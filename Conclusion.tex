\documentclass[12pt]{report}


\begin{document}
\chapter{Conclusion\label{conclusion}}
This work has successfully created a more accurate interpolation function for grain boundary (GB) energies in uranium dioxide (UO\textsubscript{2}), but revealed additional characteristics of the full five-dimensional (5D) GB space that require further research.  GB energies were found to be lower when calculated with an 800 K anneal as expected when compared to the data set created without an anneal.  Descriptions of those additional characteristics through the use of additional functions will prove beneficial.  

Future work should focus on calculating additional data points for fitting.  Increasing the number of data points will improve the quality of the fit.  Additional data points will also help to identify trends that do not readily appear with the limited data currently available. 

Further work could generalize this function to all polycrystalline materials with the fluorite crystal structure.  Such a generalization would provide further validation of this model, however, such validation would require a sufficient number of GB energies for many different materials with such a structure.  As these energies are not already available in the literature, this would require additional computational resources.  As the parameters for GB energy interpolation are improved and validated, they should be incorporated into MARMOT to further the anisotropic grain growth model being developed.

As the model develops, initial conditions set by actual nuclear fuel data will be put into a MARMOT simulation.  After simulating grain growth in a nuclear reactor, the fuel data will be compared with the simulation data to determine the accuracy of the model.
\end{document}