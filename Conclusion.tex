\documentclass[12pt]{report}


\begin{document}
\chapter{Conclusion\label{conclusion}}
This work has successfully created a more accurate interpolation function for grain boundary (GB) energies in uranium dioxide (UO\textsubscript{2}), but additional characteristics of the full five-dimensional (5D) GB space have been revealed that require further research.  GB energies were found to be lower when calculated with an 800 K anneal as expected when compared to the data set created without an anneal, but additional characteristics of the GB energy space have become apparent and require further research.  Descriptions of those additional characteristics through the use of additional functions will prove beneficial.  

The most important work yet to be done is calculation of additional data points for fitting.  Increasing the number of data points will improve the quality of the fit.  Additional data points will also help to identify trends that do not readily appear with the limited data currently available.  The difficulty associated with calculating additional data points is the required computational resources.  %Each data point (148 GB energies calculated) represents approximately one days' worth of computation time using INL's high performance computing system.  Thus, additional GB energies would take considerable time and resources. <-- This is more of a methods thing.  The second sentence might even belong more in results/discussion.

Further work could be done to generalize this function to all polycrystalline materials with the fluorite crystal structure.  Such a generalization would provide further validation of this model, however, such validation would require a sufficient number of GB energies for many different materials with such a structure.  As these energies are not already available in the literature, this would require additional computational resources.  As the parameters for GB energy interpolation are improved and validated, they should be incorporated into MARMOT to further the anisotropic grain growth model being developed.
\end{document}